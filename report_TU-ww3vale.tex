%%%%%%%%%%%%%%%%%%%%%%%%%%%%%%%%%%%%%%%%%
% University/School Laboratory Report
% LaTeX Template
% Version 3.1 (25/3/14)
%
% This template has been downloaded from:
% http://www.LaTeXTemplates.com
%
% Original author:
% Linux and Unix Users Group at Virginia Tech Wiki 
% (https://vtluug.org/wiki/Example_LaTeX_chem_lab_report)
%
% License:
% CC BY-NC-SA 3.0 (http://creativecommons.org/licenses/by-nc-sa/3.0/)
%
%%%%%%%%%%%%%%%%%%%%%%%%%%%%%%%%%%%%%%%%%
% Adaptado para relatorios automaticos de modelagem
% de ondas com o WW3
%
% Laboratorio de Insturmentacao Oceanografia
% Henrique P. P. Pereira
% Izabel C. M. Nogueira

%----------------------------------------------------------------------------------------
%	PACKAGES AND DOCUMENT CONFIGURATIONS
%----------------------------------------------------------------------------------------

\documentclass{article}

\usepackage[utf8]{inputenc} %pacote para utilizar acento
\usepackage[version=3]{mhchem} % Package for chemical equation typesetting
\usepackage{siunitx} % Provides the \SI{}{} and \si{} command for typesetting SI units
\usepackage{graphicx} % Required for the inclusion of images
\usepackage{natbib} % Required to change bibliography style to APA
\usepackage{amsmath} % Required for some math elements 
\usepackage{xcolor,colortbl} %cores para tabela
\usepackage{float} %localizacao da figura

\setlength\parindent{0pt} % Removes all indentation from paragraphs

\renewcommand{\labelenumi}{\alph{enumi}.} % Make numbering in the enumerate environment by letter rather than number (e.g. section 6)

%\usepackage{times} % Uncomment to use the Times New Roman font

%----------------------------------------------------------------------------------------
%	DOCUMENT INFORMATION
%----------------------------------------------------------------------------------------


\title{\textbf{Boletim de Previsão de Ondas - WW3} \\
\vspace{1cm}
Porto de Tubarão/ES  \\ Terminal da Ilha Guaíba/RJ} % Title

\date{\today} % Date for the report
\date{\today} % Date for the report

\begin{document}

\maketitle % Insert the title, author and date

{\large
\begin{center}
Coordenador: \\
\vspace{0.2cm}
Carlos Eduardo Parente \\
\end{center}

\vspace{0.3cm}

\begin{center}
Equipe: \\
\vspace{0.2cm}
Izabel C. M. Nogueira \\ 
Henrique P. P. Pereira \\
Isabela Cabral \\
Talitha Lourenço \\
Tamiris A. Fernandes 
\end{center}
}

\vspace{2cm}

\begin{figure}[H]
\begin{center}
\includegraphics[width=0.4\textwidth]{fig/LIOc.jpg} \\ % Include the image placeholder.png
\vspace{0.5cm}
\textbf{Laboratório de Instrumentação Oceanográfica}
\end{center}
\end{figure}

\pagebreak

%----------------------------------------------------------------------------------------
%	SECTION 1
%----------------------------------------------------------------------------------------

\section{Porto de Tubarão}


\begin{figure}[H]
\begin{center}
\includegraphics[width=1.1\textwidth]{fig/TU-local.jpg} % Include the image placeholder.png
\caption{Área de Estudo - Porto de Tubarão/ES.}
\end{center}
\end{figure}

\subsection{Resultados}

\definecolor{Gray}{gray}{0.85}

\begin{table}[H]
\renewcommand{\baselinestretch}{2}
\caption{\small Resultados da Previsão: GFS + WAVEWATCH III.}
\center{Ponto 1: Lat -20.75 Lon -40.09}
\center 
\label{tabela_periodo}
    \begin{tabular}{| c | c | c | c | c | c | c | c |}
    \hline
    \rowcolor{Gray}
    \textbf{03/12/2014} & \textbf{00Z} & \textbf{06Z} & \textbf{12Z} & \textbf{18Z} \\ \hline \hline
    \textbf{Hs} & 2.5 & 2.7 & 2.8 & 3.0 \\ \hline
    \textbf{Tp} & 12.7 & 13.0 & 13.2 & 13.5 \\ \hline
    \textbf{Dp} & 200 & 210 & 215 &  220 \\ \hline
    \textbf{Hs-Swell} & 1.5 & 1.6 & 1.4 & 1.5 \\ \hline
    \textbf{Hs-Sea} & 1 & 1.1 & 1.4 & 1.5 \\ \hline
    \hline \hline
    \rowcolor{Gray}
    \textbf{04/12/2014} & \textbf{00Z} & \textbf{06Z} & \textbf{12Z} & \textbf{18Z} \\ \hline \hline
    \textbf{Hs} & 2.5 & 2.7 & 2.8 & 3.0 \\ \hline
    \textbf{Tp} & 12.7 & 13.0 & 13.2 & 13.5 \\ \hline
    \textbf{Dp} & 200 & 210 & 215 &  220 \\ \hline
    \textbf{Hs-Swell} & 1.5 & 1.6 & 1.4 & 1.5 \\ \hline
    \textbf{Hs-Sea} & 1 & 1.1 & 1.4 & 1.5 \\ \hline
    \hline \hline
    \rowcolor{Gray}
    \textbf{05/12/2014} & \textbf{00Z} & \textbf{06Z} & \textbf{12Z} & \textbf{18Z} \\ \hline \hline
    \textbf{Hs} & 2.5 & 2.7 & 2.8 & 3.0 \\ \hline
    \textbf{Tp} & 12.7 & 13.0 & 13.2 & 13.5 \\ \hline
    \textbf{Dp} & 200 & 210 & 215 &  220 \\ \hline
    \textbf{Hs-Swell} & 1.5 & 1.6 & 1.4 & 1.5 \\ \hline
    \textbf{Hs-Sea} & 1 & 1.1 & 1.4 & 1.5 \\ \hline
    \hline \hline
    \rowcolor{Gray}
    \textbf{06/12/2014} & \textbf{00Z} & \textbf{06Z} & \textbf{12Z} & \textbf{18Z} \\ \hline \hline
    \textbf{Hs} & 2.5 & 2.7 & 2.8 & 3.0 \\ \hline
    \textbf{Tp} & 12.7 & 13.0 & 13.2 & 13.5 \\ \hline
    \textbf{Dp} & 200 & 210 & 215 &  220 \\ \hline
    \textbf{Hs-Swell} & 1.5 & 1.6 & 1.4 & 1.5 \\ \hline
    \textbf{Hs-Sea} & 1 & 1.1 & 1.4 & 1.5 \\ \hline
    \hline \hline
    \rowcolor{Gray}
    \textbf{07/12/2014} & \textbf{00Z} & \textbf{06Z} & \textbf{12Z} & \textbf{18Z} \\ \hline \hline
    \textbf{Hs} & 2.5 & 2.7 & 2.8 & 3.0 \\ \hline
    \textbf{Tp} & 12.7 & 13.0 & 13.2 & 13.5 \\ \hline
    \textbf{Dp} & 200 & 210 & 215 &  220 \\ \hline
    \textbf{Hs-Swell} & 1.5 & 1.6 & 1.4 & 1.5 \\ \hline
    \textbf{Hs-Sea} & 1 & 1.1 & 1.4 & 1.5 \\ \hline
    \hline \hline
    \rowcolor{Gray}
    \textbf{08/12/2014} & \textbf{00Z} & \textbf{06Z} & \textbf{12Z} & \textbf{18Z} \\ \hline \hline
    \textbf{Hs} & 2.5 & 2.7 & 2.8 & 3.0 \\ \hline
    \textbf{Tp} & 12.7 & 13.0 & 13.2 & 13.5 \\ \hline
    \textbf{Dp} & 200 & 210 & 215 &  220 \\ \hline
    \textbf{Hs-Swell} & 1.5 & 1.6 & 1.4 & 1.5 \\ \hline
    \textbf{Hs-Sea} & 1 & 1.1 & 1.4 & 1.5 \\ \hline
    \end{tabular}
\end{table}

\begin{figure}[t]
\begin{center}
\includegraphics[width=1\textwidth]{fig/sww3_p019.jpg} % Include the image placeholder.png
\caption{Séries temporais dos parâmetros de interesse}
\end{center}
\end{figure}

\begin{figure}[H]
\begin{center}
\includegraphics[width=1\textwidth]{fig/pspec_p019_000.jpg} % Include the image placeholder.png
\caption{Espectro de energia}
\end{center}
\end{figure}

\begin{figure}[H]
\begin{center}
\includegraphics[width=1\textwidth]{fig/dspec_p019_000.jpg} % Include the image placeholder.png
\caption{Espectro direcional}
\end{center}
\end{figure}

\begin{figure}[H]
\begin{center}
\includegraphics[width=1\textwidth]{fig/pledsww3_hs_019.jpg} % Include the image placeholder.png
\caption{Evolução do espectro direcional}
\end{center}
\end{figure}



\end{document}